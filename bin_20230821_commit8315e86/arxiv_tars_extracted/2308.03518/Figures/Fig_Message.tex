\begin{figure}[!t]
\centering
 \begin{tikzpicture} 
    \begin{axis}[
        xlabel = {$N$},
        ylabel = {MSE},
        label style={font=\footnotesize},
        width=0.48\textwidth,
        height=5cm,
        xmin=40, xmax=120,
        ymin=1e-6, ymax=.5,
        legend style={nodes={scale=0.65, transform shape}, at={(0.3,0.85)}}, 
        ticklabel style = {font=\footnotesize},
        %legend pos=west,
        ymajorgrids=true,
        xmajorgrids=true,
        grid style=dashed,
        grid=both,
        ymode = log,
        grid style={line width=.1pt, draw=gray!10},
        major grid style={line width=.2pt,draw=gray!30},
    ]
    \addplot[smooth,
             thin,
        color=chestnut,
        %mark=star,
        %mark options = {rotate = 180},
        line width=0.9pt,
        %mark size=3pt,
        ]
    table[x=N,y=x1]
    {Data/Sim3.dat};
\addplot[ smooth,
             thin,
        color=airforceblue,
        %mark=star,
        %mark options = {rotate = 180},
        line width=0.9pt,
        %mark size=3pt,
        ]
    table[x=N,y=x2]
    {Data/Sim3.dat};
    \legend{$\|\widehat{\bm{x}}_1 - \bm{x}_1\|_{2}$, $\|\widehat{\bm{x}}_2 - \bm{x}_2\|_{2}$};
    \end{axis}
\end{tikzpicture}
  \caption{ Message recovery performance versus the number of samples $N$. Here, both transmitted signals have a message of size $4$, i.e.,  $M_1 = M_2= 4$. Moreover, we consider the number of channel multipath components as $P_1 = 5$ and $P_2 = 5$. }
   \label{fig:Mas}
\end{figure}
