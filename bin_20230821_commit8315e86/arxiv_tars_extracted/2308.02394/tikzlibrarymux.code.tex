% Define default style for this node
\tikzset{every mux2 node/.style={draw,minimum width=0.6cm,minimum height=2cm,inner sep=1mm,outer sep=0pt}}
\tikzset{every mux3 node/.style={draw,minimum width=0.6cm,minimum height=3cm,inner sep=1mm,outer sep=0pt}}
\tikzset{every mux4 node/.style={draw,minimum width=0.6cm,minimum height=4cm,inner sep=1mm,outer sep=0pt}}
\tikzset{every mux5 node/.style={draw,minimum width=0.6cm,minimum height=4.8cm,inner sep=1mm,outer sep=0pt}}

\pgfdeclareshape{mux2}{

  % The 'minimum width' and 'minimum height' keys, not the content, determine
  % the size
  \savedanchor\northeast{%
    \pgfmathsetlength\pgf@x{\pgfshapeminwidth}%
    \pgfmathsetlength\pgf@y{\pgfshapeminheight}%
    \pgf@x=0.5\pgf@x
    \pgf@y=0.5\pgf@y
  }
  % This is redundant, but makes some things easier:
  \savedanchor\southwest{%
    \pgfmathsetlength\pgf@x{\pgfshapeminwidth}%
    \pgfmathsetlength\pgf@y{\pgfshapeminheight}%
    \pgf@x=-0.5\pgf@x
    \pgf@y=-0.5\pgf@y
  }
  % Inherit from rectangle
  \inheritanchorborder[from=rectangle]
  
  % Define same anchor a normal rectangle has
  \anchor{center}{\pgfpointorigin}
  \anchor{north}{\northeast \pgf@x=0pt}
  \anchor{east}{\northeast \pgf@y=0pt}
  \anchor{south}{\southwest \pgf@x=0pt}
  \anchor{west}{\southwest \pgf@y=0pt}
  \anchor{north east}{
    \pgf@process{\northeast}%
    \pgf@y=.25\pgf@y%
  }
  \anchor{north west}{\northeast \pgf@x=-\pgf@x}
  \anchor{south west}{\southwest}
  \anchor{south east}{\southwest \pgf@x=-\pgf@x}
  \anchor{text}{
    \pgfpointorigin
    \advance\pgf@x by -.5\wd\pgfnodeparttextbox%
    \advance\pgf@y by -.5\ht\pgfnodeparttextbox%
    \advance\pgf@y by +.5\dp\pgfnodeparttextbox%
  }

  \anchor{in0}{
    \pgf@process{\northeast}%
    \pgf@x=-1\pgf@x%
    \pgf@y=.5\pgf@y%
  }
  \anchor{in1}{
    \pgf@process{\northeast}%
    \pgf@x=-1\pgf@x%
    \pgf@y=-.5\pgf@y%
  }
  \anchor{sel}{
    \pgf@process{\northeast}%
    \pgf@x=0pt%
    \pgf@y=-0.7\pgf@y%
  }

  \anchor{out}{
    \pgf@process{\northeast}%
    \pgf@y=0pt%
  }

  \backgroundpath{
    \southwest \pgf@xa=\pgf@x \pgf@ya=\pgf@y
    \northeast \pgf@xb=\pgf@x \pgf@yb=\pgf@y
%
    \let\pgf@xnw=\pgf@xa
    \let\pgf@ynw=\pgf@yb
%
    \let\pgf@xsw=\pgf@xa
    \let\pgf@ysw=\pgf@ya
%
    \let\pgf@xse=\pgf@xb
    \pgf@yc=-\pgf@yb
    \advance\pgf@yc by \pgfshapeminwidth
    \let\pgf@yse=\pgf@yc
%
    \let\pgf@xne=\pgf@xb
    \pgf@xc=\pgf@yb
    \advance\pgf@xc by -\pgfshapeminwidth
    \let\pgf@yne=\pgf@xc
%
    \pgfpathmoveto{\pgfpoint{\pgf@xnw}{\pgf@ynw}}
    \pgfpathlineto{\pgfpoint{\pgf@xsw}{\pgf@ysw}}
    \pgfpathlineto{\pgfpoint{\pgf@xse}{\pgf@yse}}
    \pgfpathlineto{\pgfpoint{\pgf@xne}{\pgf@yne}}
    \pgfpathclose
    % Rectangle box
    %\pgfpathrectanglecorners{\southwest}{\northeast}
  }
}

\pgfdeclareshape{mux3}{

  % The 'minimum width' and 'minimum height' keys, not the content, determine
  % the size
  \savedanchor\northeast{%
    \pgfmathsetlength\pgf@x{\pgfshapeminwidth}%
    \pgfmathsetlength\pgf@y{\pgfshapeminheight}%
    \pgf@x=0.5\pgf@x
    \pgf@y=0.5\pgf@y
  }
  % This is redundant, but makes some things easier:
  \savedanchor\southwest{%
    \pgfmathsetlength\pgf@x{\pgfshapeminwidth}%
    \pgfmathsetlength\pgf@y{\pgfshapeminheight}%
    \pgf@x=-0.5\pgf@x
    \pgf@y=-0.5\pgf@y
  }
  % Inherit from rectangle
  \inheritanchorborder[from=rectangle]
  
  % Define same anchor a normal rectangle has
  \anchor{center}{\pgfpointorigin}
  \anchor{north}{\northeast \pgf@x=0pt}
  \anchor{east}{\northeast \pgf@y=0pt}
  \anchor{south}{\southwest \pgf@x=0pt}
  \anchor{west}{\southwest \pgf@y=0pt}
  \anchor{north east}{
    \pgf@process{\northeast}%
    \pgf@y=.25\pgf@y%
  }
  \anchor{north west}{\northeast \pgf@x=-\pgf@x}
  \anchor{south west}{\southwest}
  \anchor{south east}{\southwest \pgf@x=-\pgf@x}
  \anchor{text}{
    \pgfpointorigin
    \advance\pgf@x by -.5\wd\pgfnodeparttextbox%
    \advance\pgf@y by -.5\ht\pgfnodeparttextbox%
    \advance\pgf@y by +.5\dp\pgfnodeparttextbox%
  }

  \anchor{in0}{
    \pgf@process{\northeast}%
    \pgf@x=-1\pgf@x%
    \pgf@y=.5\pgf@y%
  }
  \anchor{in1}{
    \pgf@process{\northeast}%
    \pgf@x=-1\pgf@x%
    \pgf@y=0pt%
  }
  \anchor{in2}{
    \pgf@process{\northeast}%
    \pgf@x=-1\pgf@x%
    \pgf@y=-.5\pgf@y%
  }
  \anchor{sel}{
    \pgf@process{\northeast}%
    \pgf@x=0pt%
    \pgf@y=-0.75\pgf@y%
  }

  \anchor{out}{
    \pgf@process{\northeast}%
    \pgf@y=0pt%
  }

  \backgroundpath{
    \southwest \pgf@xa=\pgf@x \pgf@ya=\pgf@y
    \northeast \pgf@xb=\pgf@x \pgf@yb=\pgf@y
%
    \let\pgf@xnw=\pgf@xa
    \let\pgf@ynw=\pgf@yb
%
    \let\pgf@xsw=\pgf@xa
    \let\pgf@ysw=\pgf@ya
%
    \let\pgf@xse=\pgf@xb
    \pgf@yc=-\pgf@yb
    \advance\pgf@yc by \pgfshapeminwidth
    \let\pgf@yse=\pgf@yc
%
    \let\pgf@xne=\pgf@xb
    \pgf@xc=\pgf@yb
    \advance\pgf@xc by -\pgfshapeminwidth
    \let\pgf@yne=\pgf@xc
%
    \pgfpathmoveto{\pgfpoint{\pgf@xnw}{\pgf@ynw}}
    \pgfpathlineto{\pgfpoint{\pgf@xsw}{\pgf@ysw}}
    \pgfpathlineto{\pgfpoint{\pgf@xse}{\pgf@yse}}
    \pgfpathlineto{\pgfpoint{\pgf@xne}{\pgf@yne}}
    \pgfpathclose
    % Rectangle box
    %\pgfpathrectanglecorners{\southwest}{\northeast}
  }
}

\pgfdeclareshape{mux4}{

  % The 'minimum width' and 'minimum height' keys, not the content, determine
  % the size
  \savedanchor\northeast{%
    \pgfmathsetlength\pgf@x{\pgfshapeminwidth}%
    \pgfmathsetlength\pgf@y{\pgfshapeminheight}%
    \pgf@x=0.5\pgf@x
    \pgf@y=0.5\pgf@y
  }
  % This is redundant, but makes some things easier:
  \savedanchor\southwest{%
    \pgfmathsetlength\pgf@x{\pgfshapeminwidth}%
    \pgfmathsetlength\pgf@y{\pgfshapeminheight}%
    \pgf@x=-0.5\pgf@x
    \pgf@y=-0.5\pgf@y
  }
  % Inherit from rectangle
  \inheritanchorborder[from=rectangle]
  
  % Define same anchor a normal rectangle has
  \anchor{center}{\pgfpointorigin}
  \anchor{north}{\northeast \pgf@x=0pt}
  \anchor{east}{\northeast \pgf@y=0pt}
  \anchor{south}{\southwest \pgf@x=0pt}
  \anchor{west}{\southwest \pgf@y=0pt}
  \anchor{north east}{
    \pgf@process{\northeast}%
    \pgf@y=.25\pgf@y%
  }
  \anchor{north west}{\northeast \pgf@x=-\pgf@x}
  \anchor{south west}{\southwest}
  \anchor{south east}{\southwest \pgf@x=-\pgf@x}
  \anchor{text}{
    \pgfpointorigin
    \advance\pgf@x by -.5\wd\pgfnodeparttextbox%
    \advance\pgf@y by -.5\ht\pgfnodeparttextbox%
    \advance\pgf@y by +.5\dp\pgfnodeparttextbox%
  }

  \anchor{in0}{
    \pgf@process{\northeast}%
    \pgf@x=-1\pgf@x%
    \pgf@y=.6\pgf@y%
  }
  \anchor{in1}{
    \pgf@process{\northeast}%
    \pgf@x=-1\pgf@x%
    \pgf@y=0.2\pgf@y%
  }
  \anchor{in2}{
    \pgf@process{\northeast}%
    \pgf@x=-1\pgf@x%
    \pgf@y=-.2\pgf@y%
  }
  \anchor{in3}{
    \pgf@process{\northeast}%
    \pgf@x=-1\pgf@x%
    \pgf@y=-.6\pgf@y%
  }
  \anchor{sel}{
    \pgf@process{\northeast}%
    \pgf@x=0pt%
    \pgf@y=-0.75\pgf@y%
  }

  \anchor{out}{
    \pgf@process{\northeast}%
    \pgf@y=0pt%
  }

  \backgroundpath{
    \southwest \pgf@xa=\pgf@x \pgf@ya=\pgf@y
    \northeast \pgf@xb=\pgf@x \pgf@yb=\pgf@y
%
    \let\pgf@xnw=\pgf@xa
    \let\pgf@ynw=\pgf@yb
%
    \let\pgf@xsw=\pgf@xa
    \let\pgf@ysw=\pgf@ya
%
    \let\pgf@xse=\pgf@xb
    \pgf@yc=-\pgf@yb
    \advance\pgf@yc by \pgfshapeminwidth
    \let\pgf@yse=\pgf@yc
%
    \let\pgf@xne=\pgf@xb
    \pgf@xc=\pgf@yb
    \advance\pgf@xc by -\pgfshapeminwidth
    \let\pgf@yne=\pgf@xc
%
    \pgfpathmoveto{\pgfpoint{\pgf@xnw}{\pgf@ynw}}
    \pgfpathlineto{\pgfpoint{\pgf@xsw}{\pgf@ysw}}
    \pgfpathlineto{\pgfpoint{\pgf@xse}{\pgf@yse}}
    \pgfpathlineto{\pgfpoint{\pgf@xne}{\pgf@yne}}
    \pgfpathclose
  }
}

\pgfdeclareshape{mux5}{

  % The 'minimum width' and 'minimum height' keys, not the content, determine
  % the size
  \savedanchor\northeast{%
    \pgfmathsetlength\pgf@x{\pgfshapeminwidth}%
    \pgfmathsetlength\pgf@y{\pgfshapeminheight}%
    \pgf@x=0.5\pgf@x
    \pgf@y=0.5\pgf@y
  }
  % This is redundant, but makes some things easier:
  \savedanchor\southwest{%
    \pgfmathsetlength\pgf@x{\pgfshapeminwidth}%
    \pgfmathsetlength\pgf@y{\pgfshapeminheight}%
    \pgf@x=-0.5\pgf@x
    \pgf@y=-0.5\pgf@y
  }
  % Inherit from rectangle
  \inheritanchorborder[from=rectangle]
  
  % Define same anchor a normal rectangle has
  \anchor{center}{\pgfpointorigin}
  \anchor{north}{\northeast \pgf@x=0pt}
  \anchor{east}{\northeast \pgf@y=0pt}
  \anchor{south}{\southwest \pgf@x=0pt}
  \anchor{west}{\southwest \pgf@y=0pt}
  \anchor{north east}{
    \pgf@process{\northeast}%
    \pgf@y=.25\pgf@y%
  }
  \anchor{north west}{\northeast \pgf@x=-\pgf@x}
  \anchor{south west}{\southwest}
  \anchor{south east}{\southwest \pgf@x=-\pgf@x}
  \anchor{text}{
    \pgfpointorigin
    \advance\pgf@x by -.5\wd\pgfnodeparttextbox%
    \advance\pgf@y by -.5\ht\pgfnodeparttextbox%
    \advance\pgf@y by +.5\dp\pgfnodeparttextbox%
  }

  \anchor{in0}{
    \pgf@process{\northeast}%
    \pgf@x=-1\pgf@x%
    \pgf@y=.7\pgf@y%
  }
  \anchor{in1}{
    \pgf@process{\northeast}%
    \pgf@x=-1\pgf@x%
    \pgf@y=0.35\pgf@y%
  }
  \anchor{in2}{
    \pgf@process{\northeast}%
    \pgf@x=-1\pgf@x%
    \pgf@y=0\pgf@y%
  }
  \anchor{in3}{
    \pgf@process{\northeast}%
    \pgf@x=-1\pgf@x%
    \pgf@y=-.35\pgf@y%
  }
  \anchor{in4}{
    \pgf@process{\northeast}%
    \pgf@x=-1\pgf@x%
    \pgf@y=-.7\pgf@y%
  }
  \anchor{sel}{
    \pgf@process{\northeast}%
    \pgf@x=0pt%
    \pgf@y=-0.75\pgf@y%
  }

  \anchor{out}{
    \pgf@process{\northeast}%
    \pgf@y=0pt%
  }

  \backgroundpath{
    \southwest \pgf@xa=\pgf@x \pgf@ya=\pgf@y
    \northeast \pgf@xb=\pgf@x \pgf@yb=\pgf@y
%
    \let\pgf@xnw=\pgf@xa
    \let\pgf@ynw=\pgf@yb
%
    \let\pgf@xsw=\pgf@xa
    \let\pgf@ysw=\pgf@ya
%
    \let\pgf@xse=\pgf@xb
    \pgf@yc=-\pgf@yb
    \advance\pgf@yc by \pgfshapeminwidth
    \let\pgf@yse=\pgf@yc
%
    \let\pgf@xne=\pgf@xb
    \pgf@xc=\pgf@yb
    \advance\pgf@xc by -\pgfshapeminwidth
    \let\pgf@yne=\pgf@xc
%
    \pgfpathmoveto{\pgfpoint{\pgf@xnw}{\pgf@ynw}}
    \pgfpathlineto{\pgfpoint{\pgf@xsw}{\pgf@ysw}}
    \pgfpathlineto{\pgfpoint{\pgf@xse}{\pgf@yse}}
    \pgfpathlineto{\pgfpoint{\pgf@xne}{\pgf@yne}}
    \pgfpathclose
  }
}
