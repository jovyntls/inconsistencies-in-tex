
\section{Comparison of Technologies for Building \glspl{RIS}}
From a broad perspective, there are three main technologies for realizing a \gls{RIS}, namely \gls{NLC}, \gls{SC}, and \gls{MEMS}. In this section, we provide an overview of each method and discuss the respective advantages and limitations. 


\begin{table*}
\centering
\caption{Comparison of Technologies used for the Realization of RIS.}\label{tab:comparison}
\begin{tabular}{llcccc}
\hline
                  & Varactor                                 & PIN Diode                                & \gls{MEMS} Switch                 & \gls{MEMS} Mirror                  & LC                                                            \\ \hline
Tuning time       & ns                                       & ns                                       & $\mathrm{\mu s}$                                        &$\mathrm{100s \: of \: \mu s}$                                     & $>$ 10 ms                                                      \\
Tunability        & continuous                               & discrete (1 bit/diode)                   & discrete                                       & discrete                                            & continuous                                                    \\
Power consumption & medium                                   & medium                                   & low                                            & low                                                 & low                                                           \\
Scalability       & low (cost per diode)                     & low (cost per diode)                     & medium (encapsulation)                          & medium (cost per area)                              & high                                                          \\
Frequency range   & few GHz                                  & mm-Wave                                  & mm-Wave                                        & THz                                                 & $>$ 10 GHz                                                    \\
Cost              & high                                     & high                                     & high                                           & high                                                & low                                                           \\
Example           & \cite{tawk2012varactor} & \cite{tang2020wireless} & \cite{ferrari2022reconfigurable}, Ch. 4 & \cite{schmitt20213} & \cite{neuder2023compact, karabey20122} \\ \hline
\end{tabular}
\end{table*}


\subsection{\gls{NLC}-based \glspl{RIS}}
The fundamentals of \gls{NLC}-\glspl{RIS} have been discussed in detail in the previous section, hence we elaborate on the advantages and limitations of this technology. 

{\bf Advantages:} 
One of the key merits of \gls{NLC}-\glspl{RIS} is {\it scalability at low-cost}. \gls{NLC} has been used for standard \gls{LCD} fabrication processes for decades. 
Therefore, the production of large \gls{NLC} panels is technologically both very accessible and inexpensive. 
The cost of \gls{NLC}-\glspl{RIS} can be even less than \gls{LCD} displays due to the lower number of \textit{pixels}, the lack of backlight, as well as the larger tolerances compared to, for example, the strict color accuracy standards applied to displays. %To date, several start-ups are already working towards the production of \gls{NLC}-based antenna\footnote{Kymeta Corp: https://www.kymetacorp.com}. 
The other advantage of \gls{NLC} is its {\it low power consumption}.
Due to its dielectric nature, \gls{NLC} experiences minimal current flow only to alter its state, specifically for rotating the LC molecules. Nonetheless, this power requirement is significantly lower compared to other technologies such as \gls{PIN} diodes. 
The other important feature of \gls{NLC} is the {\it continuous tuning} capability, which is useful when sophisticated wavefront shaping (beyond a simple narrow reflection beam) is required, e.g., to reduce interference in multi-user systems.
%\sout{The response of the \gls{NLC} can be continuously tuned between its two extreme states, parallel and orthogonal. In the planar structures introduced in Section \ref{sc:LCRIS}, this is achieved by applying a voltage between \SI{0}{\volt} for the unbiased state, and $V_{max}$ for the fully biased states. Using common \glspl{DAC}, this can be achieved with resolutions of \SIrange{12}{16}{\bit}, achieving a quasi-continuous transition between both states. Furthermore, \gls{NLC} shows negligible hysteresis, so that a direct mapping between voltage applied and phase response is possible without considering the previous states. This technique is possible for both the reflectarray and the phased array methods, although the former often suffers from stronger amplitude variations for the different biasing states. }

{\bf Limitations:} Despite major advances in \gls{NLC} microwave community, the response time of these devices are still higher than \gls{SC} and \gls{MEMS} (e.g., $>10$~ms for \gls{PA} and {$>10$~s} for \gls{RA}). This implies that while the \glspl{RIS} are able to adapt to the user movement (e.g., on the order of seconds), they cannot adapt to fast fading. In addition, \gls{NLC}-\glspl{RIS} are suitable mainly for high frequencies $>10\ $GHz and cannot be adopted for  sub-$6$~GHz communication systems featuring rich multi-path environments. Moreover, the phase-shifting behavior of \gls{NLC}-\glspl{RIS} is temperature-dependent, see Fig.~\ref{fig:LCmolres}, which necessitates the adaption of reconfiguration strategy for the environments with significant temperature variations, see Section~\ref{sec:temperature} for possible solutions. 
%Potential reference if we want to highlight the potential of glass manufacturing without going into detail \cite{chaloun2023rf}.

\subsection{\gls{SC}-based \gls{RIS}}
\glspl{SC} are very common for \gls{PA} antennas. Similar  designs have been leveraged in prototyping \glspl{RIS}. Under this category, there are two common approaches, namely \gls{PIN} diodes and varactors. %\alejandro{AAdd FET transistors to the list? Low freq, but low consumption. It will get more complex.}

A \gls{PIN} diode is a low-capacitance device with high-frequency switching capabilities. By toggling between low and high-resistance states, the reflected wave's phase can be switched between two discrete states, typically \SI{180}{\degree} apart. Conceptually, \gls{RF}-switches are similar to \gls{PIN} diodes, as they are most commonly built as a packaged network of \glspl{PIN}. Varactors, on the other hand, have more significant capacitance variations and can be continuously tuned, but this higher capacitance also limits their maximum operation frequency.

{\bf Advantages:} \glspl{SC} are readily available at increasingly higher frequencies. Furthermore, \gls{SC}-\glspl{RIS} offer low insertion loss, fast switching speeds below a \SI{}{\micro\second}, and compact size.

{\bf Limitations:} The high power consumption (varactors, several PIN diodes) and pronounced temperature sensitivity are some of the technical disadvantages to be considered. 
However, the main factor is that the cost of building a \gls{SC}-\gls{RIS} rises quadratically with the surface area due to the increasing number of diodes required (one diode per bit and radiating element). 
%Moreover, each additional bit (doubling the number of states) requires an extra \gls{PIN} diode or an RF-switch with more ports per unit cell, which consequently at least doubles both the cost and power consumption of \gls{RIS}.
%The number of bits is not an issue with varactors since they are continuously tunable. However, the cost of varactors at mmWave frequency can be prohibitive for large scale designs.
Despite continuous advances in packaging, reliably and cost-effectively integrating thousands or even millions of discrete \gls{RF}-components in large surfaces still poses a challenge.
This remains an issue as long as the components cannot be selectively grown in the wafer, as is the case of the \glspl{TFT} used for \gls{NLC} biasing, see biasing challenges in Section~\ref{ss:biasing}.
The main limitation of using the diodes or transistors as the \gls{RF} tuning element is that they need to operate at the \gls{RIS} operating frequency, % whereas the tuning elements in \glspl{LCD} operate at very low-frequencies (sub \SI{20}{\kilo\hertz}) for biasing signals, 
which poses an extremely high demand on the processing capabilities.

% \alejandro{My idea for this section would be:

% - Solid-state / semiconductors

% - RF-Switch (shortly mention it is actually an array of semiconductor switches)

% - MEMS - I would still keep both (delay-line vs mirror) solutions, but simplify and shorten the text


% }



% \subsection{Limitations}
% We will discuss the limitations of \gls{NLC}-based \glspl{RIS} in details in Section \red{XX}.

%To date, there are X methods for building RISs, ranging from classic PIN diodes and RF-switch-based methods to more sophisticated \glspl{MEMS} and liquid crystals. 
% To paint a clearer picture of the advantages of \gls{NLC}-based \glspl{RIS}, we provide a brief overview of the most promising methods for the realization of \gls{mm-Wave} \glspl{RIS}.

%\begin{figure}[htp]
%	\centering
%	\subfloat[]{
%        \includegraphics[width=0.57\columnwidth]{Figures/technique_PIN.pdf}
%        \label{fig:technique_PIN}
%    }
%    \hfill
%	\subfloat[]{
%		\includegraphics[width=0.65\columnwidth]{Figures/technique_MEMSmirror.pdf}
%		\label{fig:technique_MEMSmirror}
%    }
%    \hfill
%	\subfloat[]{
%		\includegraphics[width=0.75\columnwidth]{Figures/technique_MEMSswitch.pdf}
%		\label{fig:technique_MEMSswitch}
%    }
%    \caption{Alternative techniques for the realization of RISs. a) PIN diodes, b) MEMS-actuated mirrors, and c) MEMS switches.}
%    \label{fig:alternatve_techniques}
%\end{figure}


% % PIN diodes limitations
% \subsection{PIN diodes/Varactors\ara{@alejandro: Can we give a name to both of them? I mean putting both in the same category }}
% \alejandro{We could group them as "semiconductor-based" solutions, maybe together with the RF-Switch (see comment in RF-Switch).}


%compared to the single \gls{PIN} diode configuration, which can be a major limiting factor at high frequencies.




% A \gls{PIN} diode is formed by the insertion of an intrinsic semiconductor layer between the p- and n-regions of a diode, as shown in Fig. \ref{fig:technique_PIN}. 
% In contrast to varactors, these diodes have negligible capacitance variation, but they can be used as switches at higher frequencies, even in the mm-Wave frequency range. 
% In forward bias, a low-resistance channel is enabled. 
% In contrast, in reverse or zero bias, a high-resistance, small capacitance is formed and the flow of the RF-field is hindered. 
% By switching between the low- and high-resistance states, the phase of the reflected wave can be switched between two discrete states, with usually a \SI{180}{\degree} difference in the reflected phase.

% \ara{Advantage:}
% widely available at Low freq., inexpensive per unit, easy to fabricate. 
% \ara{disadvantage:}
% expensive at high frequency, ... and below

% For the realization of large \glspl{RIS}, its fabrication cost must be affordable. In the case of \gls{PIN} diodes, this cost increases quadratically with the surface due to the increasing number of diodes. 
% Furthermore, the use of a single \gls{PIN} diode per unit cell enables 2 different reflection phases, typically \SI{0}{\degree} and \SI{180}{\degree}. 
% Each additional bit, i.e., doubling the number of states, requires an additional \gls{PIN} diode per unit cell, doubling both the cost and the power consumption with respect to the single \gls{PIN} diode configuration.
% As an example, the power consumption reported in \cite{tang2020wireless} at \SI{10.5}{\giga\hertz} equals \SI{0.33}{\milli\watt} per diode in an on-state. 
% A \gls{RIS} with $100 \times 100 = 10000$ elements where half of the diodes are in the on state would present a \SI{1.65}{\watt}, \SI{3.30}{\watt} or \SI{4.95}{\watt} power consumption to bias the PIN diodes in a 1-, 2-, and 3-bit configuration, respectively.
% Furthermore, an independent \gls{DC} biasing line is required for each \gls{PIN} diode in each unit cell. 
% %Higher power consumption at higher frequencies?

% \subsection{RF-Switch}

% An RF switch is a device that can be used to route RF signals from one path to another. It is typically a solid-state switch that changes its state via a DC voltage and routes the signal to a different path. In the \gls{RIS} structure, an array of RF switches are connected to the elements, which are then individually controlled to reflect or transmit the incoming RF signal. %By changing the states of the switches, the surface can be reconfigured to reflect the RF signal in a particular direction.

% {\bf Advantage.} RF-switch-based designs are easier to design, given the packaged nature of phase shifters. They also have fast switching. 

% {\bf Limitation.} Both cost and discrete nature of phase tunability are limiting factors for large-scale designs. \ara{@Alejandro: Do you mind checking the/adding to this section? Maybe power consumption?}

% \alejandro{My issue with the RF-Switch is that it is in a different, higher abstraction level than PIN, MEMS, and LC. PIN, MEMS, and LC could be used to create switches, and multiple (or an array of) switches result in an RF Switch. I think we should look at how the RF-Switches are fabricated. If it is a solid-state switch, then it is probably based on "low-frequency" PIN diodes.}


% 1-bit coding with graphene molero2021metamaterial
\subsection{MEMS-based \gls{RIS}}

From a high-level perspective, \gls{MEMS} phase shifters are structures whose electrically controlled micro-displacements result in phase shifts of the \gls{RF}-fields propagating in the component. These solutions become relevant for high-frequency systems (\gls{mm-Wave} and THz), where the wavelength is small and micro-displacements can lead to significant phase shifts. There are two different methods to use \gls{MEMS} as \gls{RF}-phase shifters: \textit{i})
\gls{MEMS}-actuated mirrors and \gls{MEMS} switches. \glspl{MEMS}-actuated mirrors that change their position in the direction normal to the surface of the antenna, varying the phase of the reflected wave; and \textit{ii)} \gls{MEMS} switches which are in principle  tunable capacitors that rely on the controlled displacement of a conductive structure inside a transmission line, often referred to as a cantilever, with an applied voltage.

\textbf{Advantages:} \gls{MEMS}-actuated mirrors have nearly negligible loss and are faster than \gls{NLC}, in the hundreds of \si{\micro\second} range. \gls{MEMS} switches share the advantage of faster tuning than \gls{NLC}. 



\textbf{Limitations:} For \gls{MEMS}-actuated mirrors, the currently achievable maximum micro-displacement of e.g. \SI{150}{\micro\meter} limit the minimum frequency \SI{1}{\tera\hertz} when $360^\circ$ phase shift is desired. 
Furthermore, their element dimension is currently larger than wavelength, thus the presence of grating lobes cannot be avoided.
To overcome the instabilities of the displacements due to the involved non-linear forces, anchoring positions are usually used, which leads to discrete displacements in practical designs.
Nevertheless, multiple-bit solutions exist such as \cite{schmitt20213} with 27 states (more than 4 bits). Finally, the fabrication and packaging is affected by the same cost limitation as \glspl{SC}. 
Similar to actuated mirrors, the stability of \gls{MEMS} switches limits the number of phase-shift states. In addition, the insertion loss of \gls{MEMS} switches is in a similar range as \glspl{NLC} for the lower \gls{mm-Wave} frequencies and increases with frequency.

A comparative summary of the advantages and limitations of the discussed \gls{RIS} technologies are presented in Tab~\ref{tab:comparison}.
%{\bf Advantages.} 

%{\bf Limitations.}\ara{Can you please add the limitation of both here. Hard to maintain mechanical stability?and the perhaps 1 advantage per design}


%\ara{@AJS: I tried to extract information from the text below but it was hard for me to identify which one is a joint advantage/limitaiton and what is not. I think you are in the best position to take care of this. }
% MEMS - Add fig
%Such an approach minimizes losses, since the tunability is achieved by the reflection of a wave in a highly reflective and low-loss surface. 
%Furthermore, the use of voltage-actuated mirrors should reduce the power consumption of large panels compared to \gls{PIN} diodes, since no current is needed to keep the state, but only for the mechanical displacement of a lightweight mass when switching states. % At the same time, the current control of the mirrors is discrete and in the range of 3-bit, so that the reflected phase could be tuned in \SI{45}{\degree} steps and not continuously.

%The discrete phase shifts of the reflected wave are achieved by mechanically fixing the structure at certain positions. An advantage of this approach is that the biasing can be completely removed after the displacement.
%To cope with hystheresis and stability concerns, such structures are designed to achieve discrete states. For example, in \cite{schmitt20213}, 27 states have been achieved by using mechanical amplifiers, ideally achieving \SI{13.3}{\degree} steps with a tuning voltage of around \SI{60}{\volt} and a maximum displacement of \SI{149.5}{\micro\meter}.
%To cover a full \SI{360}{\degree} phase shift of the reflected wave, the mirrors need to be displaced up to half a wavelength. With a \SI{149.5}{\micro\meter} displacement, this is currently achieved for frequencies equal or higher than \SI{1}{\tera\hertz}.
%In addition, the performance of the \glspl{RIS} will be defined by the ability to further shrink the area of such actuators towards $0.5\lambda_0 \times 0.5\lambda_0$.
%Concerning the cost, the price will be highly influenced by the silicon cost and, if required, the process of metalizing the reflector without affecting the sensitive actuator behind the mirror. 
%With the \gls{DRIE} fabrication process, several-inches wide wafers can be mass-produced, and a posterior efficient and reliable modular assembly of several modules is needed to realize large \gls{RIS} panels.

%\alejandro{
%Advantage:
%Negligible loss, 
%faster than LC, 100 µs range

%Disadvantage (will need to prioritize):
%Only for high frequencies;
%Currently large mirrors, so no full tuning
%Only phase, no amplitude tuning;
%Response time of hundreds of µs/ some ms;
%Mechanical oscillations appear when changing state;
%Assembly is complex. Very sensitive to any electrostatic charge (for example, dust);

%Other remarks:
%Very simple from the HF perspective, very complex from the mechanical one;

%Todo: Shorten text, prioritize a few advantages and disadvantages.
%}


%\gls{MEMS} switches could in principle be continuously tunable capacitors by relying on the continuous displacement of a conductive structure, often referred to as a cantilever, with an applied voltage as shown in Fig. \ref{fig:technique_MEMSswitch}. In this case, the voltage needs to be held to maintain the displacement and the structure returns to its original position when the biasing is removed. However, the realization of such analog \gls{MEMS} varactors for commercial applications is not possible due to strong performance variations as a result of device tolerances \cite{rehder2010low, ferrari2022reconfigurable}. Similarly as with the mirrors, a useful approach is using the device as a switch and cascade several \gls{MEMS} switches to realize multi-bit discrete phase shifters. By doing this, 2- and 3-bit discrete mm-Wave phase shifters with $FoM$ up to \SI{99}{\degree/dB} have been reported \cite{hung2004distributed}.

%\alejandro{
%Advantage
%faster with a comparable performance to LC

%Disadvantage
%worse for higher frequencies $>$ 100 GHz?
%More complex
%Stability issues, therefore not commercially available.

%Fazit: Just mention them very briefly and concentrate on the first type?
%}


% \sout{Further technologies are the use of mechanical \gls{RF}-switches, as well as the use of varactors. However, neither of these technologies is scalable towards mm-Wave and even THz systems and are hence not further explained in this work.
% Limitations limited bits
% Assuming a single reflected beam with maximum gain is desired, the use of discrete phase shifters with 1, 2, and 3 bits results in gain reductions of 3.9, 0.9, and \SI{0.2}{dB}, respectively \cite{wu2019beamforming}. 
% For more challenging systems such as multiple transmitters and/or receivers, an even higher performance deterioration occurs when relying on discrete phase values. 
% Compared to the \SIrange{1}{3}{\bit} from \gls{PIN} diode solutions and the \SIrange{3}{5}{bit} of the novel solutions based on \gls{MEMS} mirrors, the tuning resolution of \gls{NLC} \glspl{RIS} is only limited by the \gls{DAC}.
% Commercial \glspl{DAC} with \SIrange{12}{16}{bits} are widely available. % at affordable prices well below US\$50.
%\ara{Maybe a bit too compressed? but still long :D. Does it make sene to follow a structure for all the three technologies: a few lines about how they work. then a few line about pros and a few about cons. I do like the fact that we provide numbers btw}
%}
% \ara{We can also add RF-switch?}




