\section{Conclusion}
\label{sc:conclusion}

This paper presented \gls{NLC} as an enabler technology for building extremely large \glspl{RIS} with continuous tuning capability, low power consumption, frequency scalability, and cost-efficient fabrication. However, these advantages come with the limitations such as slow response time and temperature dependencies. The basic physical principles of \gls{NLC} theory were reviewed and two important implementations of \gls{NLC}-\glspl{RIS} were introduced. Finally, \gls{NLC}-\glspl{RIS} were compared against competing technologies \gls{SC}- and \gls{MEMS}-\glspl{RIS} and several important open research problems on  \gls{NLC}-\glspl{RIS} were presented. 
