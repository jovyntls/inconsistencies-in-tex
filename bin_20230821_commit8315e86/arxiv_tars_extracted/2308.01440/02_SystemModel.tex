\section{System Model}\label{System-Model}


The cellular network under consideration is depicted in Fig.~\ref{fig:illustration} and detailed as follows.


\subsection{Network Topology}\label{Network-Topology}


\subsubsection{Ground Cellular Network} The underlying infrastructure of our network is a terrestrial cellular deployment consisting of $N$ BSs that provide service to network users. The height and $2$D location of BS $n$ is denoted by $h_{n,\mathrm{B}}$ and $\bm{p}_n$, respectively, for each $n \in \{1, \cdots, N\}$. Let $\bm{\Theta} = (\theta_1, \cdots, \theta_N)$ where $\theta_n \in [-90^\circ, +90^\circ]$ is the vertical antenna tilt of BS $n$, that can be adjusted by a mobile operator, with positive and negative angles denoting uptilts and downtilts, respectively. Let $\bm{\rho} = (\rho_1, \cdots, \rho_N)$ where $\rho_n$ is the transmission power of BS $n$, measured in dBm, which is also adjustable by a mobile operator with a maximum value of $\rho_{\max}$. We denote the antenna horizontal boresight direction (azimuth) of BS $n$ by $\phi_n \in [-180^\circ, +180^\circ]$ which is assumed to be fixed upon deployment. 


\subsubsection{UAV Corridors and Legacy Ground Users} There are two types of users being served by the BSs: (i) UAVs that traverse a region $Q_U = \bigcup_{u=1}^{N_U} Q_u$ consisting of $N_U$ predefined 2D aerial routes/corridors $Q_u$; and (ii) ground-users (GUEs) that are dispersed over a 2D region $Q_G$. For each $1 \leq u \leq N_U$, all UAVs flying in the aerial corridor $Q_u$ are assumed to have a fixed height $h_u$. In addition, we consider a fixed height $h_G$ for all GUEs. Let $\lambda(\bm{q})$ be a probability density function that represents the distribution of users in the target region $Q = Q_U \bigcup Q_G$. Each user is associated with one BS; thus, the target region $Q$ is partitioned into $N$ disjoint subregions $\bm{V} = (V_1, \cdots, V_N)$ such that users within $V_n$ are associated with BS $n$.



\subsection{Channel Model}\label{Channel-Model}


\subsubsection{Antenna Gain} The BSs use  directional antennas with vertical and horizontal half-power beamwidths of $\theta_{\textrm{3dB}}$ and $\phi_{\textrm{3dB}}$, respectively. Let $A_{\max}$ be the maximum antenna gain at the boresight and denote the vertical and horizontal antenna gains in dB by $A_{n,\bm{q}}^{V}$ and $A_{n,\bm{q}}^{H}$, respectively. Directional antenna gains are given by \cite{3GPP38901}:
\begin{equation}\label{directional-antenna-gains}
    A_{n,\bm{q}}^{\mathrm{V}} =
    - \frac{12}{\theta^2_{\text{3dB}}} \left[ \theta_{n,\bm{q}} - \theta_n \right]^2, \quad
    A_{n,\bm{q}}^{\mathrm{H}} =
    - \frac{12}{\phi^2_{\text{3dB}}} \left[ \phi_{n,\bm{q}} - \phi_n \right]^2, 
\end{equation}
where $\theta_{n,\bm{q}}$ and $\phi_{n,\bm{q}}$ are the elevation angle and the azimuth angle between BS $n$ and the user location $\bm{q} \in Q$, respectively. These angles can be calculated as:
\begin{align}\label{theta-phi-nq}
    \theta_{n,\bm{q}} &= \tan^{-1}\!\left( \frac{h_{\bm{q}} - h_{n,\mathrm{B}}}{\| \bm{q} - \bm{p}_n \|}  \right), \\
    \phi_{n,\bm{q}} &= \!
\begin{cases}
\tan^{-1}\!\left(\frac{q_{\mathrm{y}}-p_{n,\mathrm{y}}}{q_{\mathrm{x}}-p_{n,\mathrm{x}}}\right) \!+\! 180^{\circ}\!\times\! 2c &  \!\!\!\text{if $q_{\mathrm{x}}-p_{n,\mathrm{x}}>0$}\\
\tan^{-1}\!\left(\frac{q_{\mathrm{y}}-p_{n,\mathrm{y}}}{q_{\mathrm{x}}-p_{n,\mathrm{x}}}\right) \!+\! 180^{\circ}\!\times\! (2c + 1) &  \!\!\!\text{if $q_{\mathrm{x}}-p_{n,\mathrm{x}}<0$}
\end{cases}\nonumber
\end{align}
where subscripts $\cdot_{\mathrm{x}}$ and $\cdot_{\mathrm{y}}$ denote the horizontal and vertical Cartesian coordinates of a point, respectively, and the integer $c$ is selected such that $-180^{\circ} \leq \phi_{n,\bm{q}} - \phi_n \leq +180^{\circ}$. Thus, the total antenna gain of BS $n$ in dB is given by $A_{n, \bm{q}} = A_{\max} + A_{n,\bm{q}}^{\mathrm{V}} +A_{n,\bm{q}}^{\mathrm{H}}$.

% \begin{equation}\label{total-antenna-gain}
%     A_{n, \bm{q}} = A_{\max} + A_{n,\bm{q}}^{\mathrm{V}} +A_{n,\bm{q}}^{\mathrm{H}}.
% \end{equation}


\subsubsection{Pathloss} The pathloss $L_{n,\bm{q}}$ between BS $n$ and the user location $\bm{q}$ is a function of their distance and given by:
\begin{equation} \label{eqn:Pathloss}
L_{n,\bm{q}} = a_{\bm{q}} + b_{\bm{q}} \log_{10}\left[\| \bm{q} - \bm{p}_n \|^2 + (h_{\bm{q}} -h_{n,\mathrm{B}})^2 \right]^{\frac{1}{2}},
\end{equation}
where $a_{\bm{q}}$ depends on the carrier frequency and $b_{\bm{q}}$ relates to the line-of-sight condition and the pathloss exponent, which depends on the BS deployment feature and the user height at $\bm{q}$. In our case study in Section \ref{case-study}, we utilize practical values for the constants $a_{\bm{q}}$ and $b_{\bm{q}}$ that are adopted from the 3GPP studies \cite{3GPP36777,3GPP38901}.

In the remainder of the manuscript, we assume that the BS location $\bm{p}_n$ and the azimuth orientation $\phi_n$ are fixed for all $n \in \{1, \cdots, N\}$. We optimize over the vertical antenna tilts, cell partitioning, and BS transmission powers for multiple performance metrics that are introduced in the next two sections.