\section{Conclusion}\label{conclusion}


In this paper, we took the first step towards creating a mathematical framework for optimizing antenna tilts and transmit power in cellular networks, with the goal of providing the best quality of service to both legacy ground users and UAVs flying along corridors. By applying quantization theory and designing iterative algorithms, we modeled realistic features of network deployment, antenna radiation patterns, and propagation channel models. Our proposed algorithms offer the capability to optimize coverage and signal quality %, and fairness,
while allowing for trade-offs between performance on the ground and along UAV corridors through adjustable hyperparameters. The optimal combinations of antenna tilts and transmit power, which are non-obvious and challenging to design, were shown to significantly enhance performance along UAV corridors. Importantly, these improvements come at a negligible-to-moderate sacrifice in ground user performance compared to scenarios without UAVs.

To the best of our knowledge, this is the first work that determines the necessary conditions and designs iterative algorithms to optimize cellular networks for UAV corridors using quantization theory. Our findings open avenues for further exploration and extensions from multiple standpoints, some of which are listed as follows: (i) Performance metric, optimizing for capacity per user, rather than SINR, thus aligning more closely with the objectives of real-world mobile network operators; 
(ii) Antenna pattern, considering BSs transmitting multiple beamformed synchronization signal blocks (SSBs), instead of a single beam, and addressing the optimization of the SSB codebooks;
(iii) Cellular deployment, exploring the optimization of BS locations, in addition to their antenna tilts and transmit power; and
(iv) Channel model, replacing the statistical 3GPP model with a scenario-specific map-based channel model, providing a more accurate, ad-hoc representation of the channel characteristics. 
Progress along any of the above directions would extend the applicability and scope of our work, paving the way for advancements in optimizing cellular networks for UAV corridors and addressing emerging challenges in air-to-ground wireless communications.