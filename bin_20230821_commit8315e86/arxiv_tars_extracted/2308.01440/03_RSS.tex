\section{Optimal RSS in Cellular Networks}\label{RSS-Chapter}

\subsection{Problem Formulation}\label{RSS-Problem-Formulation}

Our aim in this section is to optimize the average RSS across all users within the target region $Q$. The RSS from BS $n$, measured in dBm, provided at the user location $\bm{q}$ is given by:
\begin{multline}\label{RSS-dBm}
\mathtt{RSS_{dBm}^{(n)}}(\bm{q}; \theta_n)= \rho_n + A_{n,\bm{q}} - L_{n,\bm{q}}
 = \rho_n + A_{\textrm{max}} \\ - \frac{12}{\theta^2_{\text{3dB}}} \left[ \theta_{n,\bm{q}} - \theta_n \right]^2 - \frac{12}{\phi^2_{\text{3dB}}} \left[ \phi_{n,\bm{q}} - \phi_n \right]^2 
  - a_{\bm{q}} \\ - b_{\bm{q}} \log_{10}\left[\| \bm{q} - \bm{p}_n \|^2 + (h_{\bm{q}} -h_{n,\mathrm{B}})^2 \right]^{\frac{1}{2}}.
\end{multline}
The overall performance function, i.e., the RSS averaged over all network users, is given by:
\begin{equation}\label{RSS-objective-function}
    \Phi_{\mathtt{RSS}}(\bm{V}, \bm{\Theta}) = \sum_{n=1}^{N} \int_{V_n} \mathtt{RSS_{dBm}^{(n)}}(\bm{q}; \bm{\Theta}) \lambda(\bm{q}) d\bm{q}.
\end{equation}
In what follows, we seek to maximize the performance function $\Phi_{\mathtt{RSS}}$ in Eq. (\ref{RSS-objective-function}) over the cell partitioning $\bm{V}$ and BS vertical antenna tilts $\bm{\Theta}$.

\begin{Remark}
Due to the absence of interference from neighboring cells, optimizing $\Phi_{\mathtt{RSS}}$ w.r.t. the BS transmission powers $\bm{\rho}$ always reduces to allocating the maximum transmission power $\rho_{\max}$ to each BS. Hence, we only optimize over the cell partitioning and BS vertical antenna tilts while assuming that the BS transmission powers $\bm{\rho}$ are given and fixed.
\end{Remark}

\begin{Remark}
The RSS function in Eq. (\ref{RSS-dBm}) is not necessarily a non-increasing function of the distance between the BS and the user. This is because while moving away from the BS worsens the pathloss component, it may lead to a better antenna gain and thus, an overall RSS value.
\end{Remark}




\subsection{Analytical Framework}\label{RSS-Analytical-Framework}

Our goal is to optimize the performance function $\Phi_{\mathtt{RSS}}$ over variables $\bm{V}$ and $\bm{\Theta}$. Not only does the optimal choice of each variable depend on the value of the other, but also this optimization problem is NP-hard. Our approach is to design an alternating optimization algorithm that iterates between updating $\bm{V}$ and $\bm{\Theta}$. 
In quantization theory, variations of the Lloyd algorithm \cite{lloyd1982least, gray1998quantization} have been used to solve similar optimization problems. Inspired by quantization theory, we need to: (i) find the optimal cell partitioning $\bm{V}$ given a set of BS vertical antenna tilts $\bm{\Theta}$; and (ii) find the optimal vertical antenna tilts $\bm{\Theta}$ for a given cell partitioning $\bm{V}$. The solution of the first task is a generalized Voronoi tessellation \cite{boots2009spatial, du1999centroidal} carried out via the following proposition:
\begin{Proposition}\label{optimal-V}
For a given set of BS vertical antenna tilts $\bm{\Theta}$, the optimal cell partitioning $\bm{V}^*(\bm{\Theta}) = \big(V_1^*(\bm{\Theta}), \cdots, V_N^*(\bm{\Theta})\big)$ that maximizes the performance function $\Phi_{\mathtt{RSS}}$ is given by:
\begin{multline}\label{optimal-cell-partitioning}
    V_n^*(\bm{\Theta}) = \big\{\bm{q} \in Q \mid \mathtt{RSS_{dBm}^{(n)}}(\bm{q}; \theta_n) \geq \mathtt{RSS_{dBm}^{(k)}}(\bm{q}; \theta_k), \\ \textrm{ for all } 1 \leq k \leq N \big\},
\end{multline}
for each $n \in \{1, \cdots, N\}$. The ties can be broken arbitrarily.
\end{Proposition}
\textit{Proof. }Let $\bm{W} = (W_1, \cdots, W_N)$ be any arbitrary cell partitioning of the target region $Q$. Then: 
\begin{align*}
    \Phi_{\mathtt{dBm}}(\bm{W},\bm{\Theta}) &= \sum_{n=1}^{N} \int_{W_n} \mathtt{RSS_{dBm}^{(n)}} (\bm{q}; \bm{\Theta}) \lambda(\bm{q}) d\bm{q} \\& \leq \sum_{n=1}^{N} \int_{W_n} \max_k \Big[\mathtt{RSS_{dBm}^{(k)}} (\bm{q}; \bm{\Theta})\Big] \lambda(\bm{q}) d\bm{q} \\
    &=\int_Q \max_k \Big[\mathtt{RSS_{dBm}^{(k)}} (\bm{q}; \bm{\Theta})\Big] \lambda(\bm{q}) d\bm{q} \\&=\sum_{n=1}^{N} \int_{V_n^*} \max_k \Big[\mathtt{RSS_{dBm}^{(k)}} (\bm{q}; \bm{\Theta}))\Big] \lambda(\bm{q}) d\bm{q} \\
    &=\sum_{n=1}^{N} \int_{V_n^*} \mathtt{RSS_{dBm}^{(n)}} (\bm{q}; \bm{\Theta}) \lambda(\bm{q}) d\bm{q} \\&= \Phi_{\mathtt{dBm}}(\bm{V}^*, \bm{\Theta}),
\end{align*}
i.e., $\bm{V}^*$ achieves a performance no less than any other partitioning $\bm{W}$ and is optimal.$\hfill\blacksquare$


For the second task, our approach is to apply gradient ascent to find the optimal BS vertical antenna tilts $\bm{\Theta}$ for a given cell partitioning $\bm{V}$. Gradient ascent is a first-order optimization algorithm that iteratively refines the estimate of a locally optimal $\bm{\Theta}$ by following the direction of the gradient.
\begin{Proposition}\label{rss-grad-eq}
The partial derivative of the performance function $\Phi_{\mathtt{RSS}}$ w.r.t. $\theta_n$ is given by:
\begin{align}\label{eqn:derivativePhidBm}
\frac{\partial \Phi(\bm{V},\bm{\Theta})}{\partial \theta_n}  =   \frac{24}{\theta^2_{\text{3dB}}} \Bigg\{ \sum_{u=1}^{N_U}
      \int_{V_n(\mathbf{\Theta})\cap Q_u} \!\!\! (\theta_{n,\bm{q}}-\theta_n) \lambda(\bm{q}) d\bm{q}   \nonumber\\+ \int_{V_n(\mathbf{\Theta})\cap Q_G} \!\!\! (\theta_{n,\bm{q}}-\theta_n) \lambda(\bm{q}) d\bm{q} \Bigg\}.
\end{align}
\end{Proposition}
\textit{Proof. }The partial derivative of Eq. (\ref{RSS-objective-function}) w.r.t. $\theta_n$ consists of two components: (i) the derivative of the integrand; and (ii) the integral over the boundaries of $V_n$ and its neighboring regions. For any point $\bm{q}$ on the boundary of neighboring regions $V_n$ and $V_m$, the normal outward vectors have opposite directions and we have $\mathtt{RSS_{dBm}^{(n)}}(\bm{q}; \bm{\Theta}) = \mathtt{RSS_{dBm}^{(m)}}(\bm{q}; \bm{\Theta})$; thus, the sum of elements in the second component is zero \cite{GuoJaf2016}. The first component evaluates to:
\begin{multline}\label{eqn:derivativePhi}
    \frac{\partial \Phi(\mathbf{V},\mathbf{\Theta})}{\partial \theta_n} = \int_{V_n(\mathbf{\Theta})} \frac{\partial}{\partial \theta_n} \mathtt{RSS_{dBm}^{(n)}}(\bm{q}; \theta_n) \lambda(\bm{q})d\bm{q}  \\
    \stackrel{(\text{a})}{=}  \frac{24}{\theta^2_{\text{3dB}}} \Bigg\{ \sum_{u=1}^{N_U}
      \int_{V_n(\mathbf{\Theta})\cap Q_u}  (\theta_{n,\bm{q}}-\theta_n) \lambda(\bm{q}) d\bm{q}   
      \\+  \int_{V_n(\mathbf{\Theta})\cap Q_G}  (\theta_{n,\bm{q}}-\theta_n) \lambda(\bm{q}) d\bm{q} \Bigg\},
\end{multline}
where (a) follows from the definition of $Q = Q_U \bigcup Q_G$, and the proof is complete. $\hfill\blacksquare$


\subsection{Proposed Algorithm}\label{RSS-Algorithm}

With our two tasks accomplished in Propositions \ref{optimal-V} and \ref{rss-grad-eq}, we propose the maximum-RSS vertical antenna tilt (Max-RSS-VAT) iterative optimization algorithm outlined in Algorithm \ref{BS_VAT_Algorithm}.

\begin{algorithm}[ht!]
\SetAlgoLined
\SetKwRepeat{Do}{do}{while}
\KwResult{Optimal BS vertical antenna tilts $\bm{\Theta}^*$ and cell partitioning $\bm{V}^*$.}
\textbf{Input:} Initial BS vertical antenna tilts $\mathbf{\Theta}$ and cell partitioning $\bm{V}$, learning rate $\eta_0\in (0,1)$, convergence error thresholds $\epsilon_1, \epsilon_2\in \mathbb{R}^+$, constant $\kappa \in (0, 1)$\;

\Do{$\frac{\Phi_{\mathtt{RSS}}^{\textrm{(new)}} - \Phi_{\mathtt{RSS}}^{\textrm{(old)}}}{\Phi_{\mathtt{RSS}}^{\textrm{(old)}}} \geq \epsilon_2$}
{
-- Calculate  $\Phi_{\mathtt{RSS}}^{\textrm{(old)}} = \Phi_{\mathtt{RSS}}\left(\bm{V},\mathbf{\Theta}\right)$\;
-- Update the cell $V_n$ according to Eq. (\ref{optimal-cell-partitioning}) for each $n \in \{1, \cdots, N\}$\;
-- Set $\eta \gets \eta_0$\;
\Do{$\frac{\Phi_{\textrm{e}} - \Phi_{\textrm{s}}}{\Phi_{\textrm{s}}} \geq \epsilon_1$}
{
-- Calculate  $\Phi_{\textrm{s}} = \Phi_{\mathtt{RSS}}\left(\bm{V},\mathbf{\Theta}\right)$\;
-- Calculate $\frac{\partial \Phi_{\mathtt{RSS}}(\mathbf{V},\mathbf{\Theta})}{\partial \theta_n}$ according to Eq. (\ref{eqn:derivativePhidBm}) for each $n \in \{1, \cdots, N\}$\;
-- $\eta \gets \eta \times \kappa$\;
-- $\mathbf{\Theta} \gets \mathbf{\Theta} + \eta \nabla_{\mathbf{\Theta}} \Phi_{\mathtt{RSS}}(\bm{V},\mathbf{\Theta})$\;
-- Calculate $\Phi_{\textrm{e}} = \Phi_{\mathtt{RSS}}\left(\bm{V},\mathbf{\Theta}\right)$\;
}
-- Calculate  $\Phi_{\mathtt{RSS}}^{\textrm{(new)}} = \Phi_{\mathtt{RSS}}\left(\bm{V},\mathbf{\Theta}\right)$\;
}
 \caption{Maximum-RSS vertical antenna tilt (Max-RSS-VAT) optimization}
 \label{BS_VAT_Algorithm}
\end{algorithm}

\begin{Proposition}\label{BS-VAT-convergence}
The Max-RSS-VAT algorithm is an iterative improvement algorithm and converges.
\end{Proposition}
\textit{Proof. }Proposition \ref{optimal-V} indicates that updating the cell $V_n$ according to Eq. (\ref{optimal-cell-partitioning}), as it is done in the Max-RSS-VAT algorithm, yields the optimal cell partitioning for a given value of $\bm{\Theta}$; thus, the performance function $\Phi_{\mathtt{RSS}}$ will not decrease as a result of this update rule. The Max-RSS-VAT algorithm updates the vertical antenna tilts $\bm{\Theta}$ using gradient ascent where the learning rate at time $t$ is given by $\eta_t = \eta_0 \times \kappa^t$. Because $\sum_{t=1}^{\infty}\eta^2_t < \sum_{t=1}^{\infty}\eta_t = \frac{\kappa}{1 - \kappa}\eta_0 < \infty$, the gradient ascent is guaranteed to converge \cite{goodfellow2016deep} and does not decrease the performance function $\Phi_{\mathtt{RSS}}$. Hence, the Max-RSS-VAT algorithm generates a sequence of non-decreasing performance function values that are also upper bounded because of the limited transmission power at each BS; thus, the algorithm converges.  $\hfill\blacksquare$